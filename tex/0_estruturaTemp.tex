\section{Tema}%
ANÁLISE DAS EVASÕES DE ALUNOS DOS CURSOS DA UnB: Um estudo no âmbito da graduação e pós-graduação

\section{Problema}%
As altas taxas de evasão de cursos de nível superior no Brasil é um problema a ser estudado e combatido pelas instituições de nível superior. Existem cursos onde o  número de concluintes é absolutamente inferior aos de ingressantes. 

Conseguir prever o desempenho dos alunos durante a vida acadêmica é um desafio para as equipes de planejamento de graduação. Políticas sociais são lançadas e as mesmas devem ser avaliadas para não comprometer o nível intelectual da Universidade.

A necessidade de explorar dados para prever tendências, descobrir padrões e extrair inteligência de dados de um determinado negócio, ou seja, transformar dados em informação é um processo que vem sendo aplicado com  maior frequência pelas grandes instituições para apoio as tomadas de decisões e definição das diretrizes por parte das equipes de planejamento das mesmas.

A intenção do discente em evadir às vezes é conhecida tardiamente impedindo que a gestão da Universidade tome providências para reverter o quadro.

Ao conseguir identificar o risco de um aluno fracassar, ou seja, evadir, com antecedência pode ajudar os gestores da Universidade a elaborar um plano personalizado para mitigação deste risco.
Para uma melhor compreensão das evasões podemos trabalhar com questionamentos , tais como:

\begin{enumerate} 
    \begin{itemize}
        \item	Cursos da área de exatas possuem maior índice de evasões comparados aos da área de humanas?
        \item	Distância entre Campi e a residência do discente influência na decisão para evadir?
        \item	Há correlação entre evasões e ingresso pelo programa de cotas?
        \item	Greve do restaurante Universitário influência a evasão de discentes baixa renda?
    \end{itemize}%
\end{enumerate}%

Visto o exposto, a administração das Universidades podem ser mais eficientes para combater este problema conhecendo melhor os motivos pelos quais levam os discentes a desistirem dos cursos.  


\section{Objetivos}%

\subsection{Objetivo Geral}
O objetivo da proposta é um estudo acerca da evasão dos cursos ofertados pela Universidade de Brasília. Comparar os índices de evasões entre os cursos relacionados a dados acadêmicos e pessoais dos alunos com o intuito  de encontrar padrões que possam auxiliar a identificação de possíveis evasões com antecedência. Permitindo assim um trabalho de mitigação dos riscos de fracasso dos mesmos, ou seja, o aluno sair do curso por um motivo diferente de formatura.
 
Para realização do estudo serão usadas como fonte as bases de dados dos sistemas legados da Universidade de Brasília, com enfoque no ciclo de vida acadêmico dos discentes e docentes. Aplicar técnicas de mineração de dados para a descoberta de novos padrões relacionados as diversas formas de evasões dos cursos ofertados pela UnB para que possam prever tendências que  auxiliarão a equipe de planejamento a acompanhar e definir as metas e diretrizes da Universidade no âmbito acadêmico.

Ao identificar se um aluno de um determinado curso corre o risco de fracassar, por exemplo, abandonar o curso, com antecedência é possível tomar providências cabíveis para mitigar ou até mesmo eliminar este risco. Agindo com antecedência é possível elaborar um plano de intervenção personalizado para melhoria do desempenho evitando o abandono e consequentemente o desperdício de dinheiro público no caso da evasão do mesmo.

\subsection{Objetivo específicos}
\begin{enumerate} 
    \begin{itemize}
        \item	Apresentar a informação de maneira consistente e única;
        \item	Fomentar as áreas de negócio com informações do estudo para que tomem decisões mais precisas;
        \item	Padronizar a informação de diversas fontes;
        \item	Analisar os dados de um determinado período;
        \item   Comparar a evasão entre os cursos da UnB;
        \item   Conhecimento dos dados gerados pelos sistemas acadêmicos legados com enfoque nas evasões.
    \end{itemize}%
\end{enumerate}%
\section{Citações}%
   Citações: 
   \cite{Gilioli2016EVASAOGilioli},%
   \cite{Catani2006PROUNI:Superior}
   \cite{Andriola2006OpinioesUFC}
   \cite{Beatriz2007ABrasileiro}
   \cite{Lobo2012PanoramaSolucoes}
   \cite{Nassar2011Do.}
   \cite{Pereira2003DeterminantesPara}
   \cite{Silva2014EvasaoTransacao}
   \cite{Veneroso2016AsaaFederal}
   \cite{Vitelli2010EvasaoFenomeno}
   \cite{Pereira2011EvasaoGrosso}

  

