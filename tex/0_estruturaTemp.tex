\section{Tema}%
ANÁLISE DA EVASÃO DE ALUNOS DOS CURSOS DA UnB: Um estudo no âmbito da graduação

\section{Contextualização}%

As altas taxas de evasão de cursos de nível superior no Brasil é um problema a ser estudado e combatido pelas instituições de nível superior. Existem cursos onde o  número de concluintes é absolutamente inferior aos de ingressantes.

 “A evasão estudantil no ensino superior é um problema internacional que afeta o resultado dos sistemas educacionais. As perdas de estudantes que iniciam mas não terminam seus cursos são desperdícios sociais, acadêmicos e econômicos. No setor público, são recursos públicos investidos sem o devido retorno.” , diz [Silva Filho,2007].%\cite{SilvaFilhoRobertoLealLoboandMotejunasPauloRobertoandHipolitoOscarandLobo2007ABrasileiro.} 

Segundo a reportagem do sítio G1 com o Sindicato das Entidades Mantenedoras de Estabelecimentos de Ensino Superior no Estado de São Paulo (SEMESP) [G1,2012] %\cite{SindicatodasEntidadesMantenedorasdeEstabelecimentosdeEnsinoSuperiornoEstadodeSaoPauloSEMESP2012IndiceInformacao}
, “A cada três alunos que entram no curso de sistemas de informação, apenas um recebe o diploma.  Em ciência da computação, a cada quatro alunos que entram no curso, apenas um termina. O que acontece é que muitos estudantes são atraídos para o mercado de trabalho, que está aquecido, mas sofre com a falta de mão de obra.” . 

Conseguir prever o desempenho dos alunos durante a vida acadêmica é um desafio para as equipes de planejamento de graduação. 

A necessidade de explorar dados para prever tendências, descobrir padrões e extrair inteligência de dados de um determinado negócio, ou seja, transformar dados em informação é um processo que vem sendo aplicado com  maior frequência pelas grandes instituições para apoio às tomadas de decisões e à definição das diretrizes por parte das equipes de planejamento das mesmas.

A intenção do discente em evadir às vezes é conhecida tardiamente impedindo que a gestão da Universidade tome providências para reverter o quadro.

Ao conseguir identificar o risco de um aluno fracassar, ou seja, evadir, com antecedência pode ajudar os gestores da Universidade a elaborar um plano personalizado para mitigação deste risco.
%Para uma melhor compreensão das evasões podemos trabalhar com questionamentos , tais como:

\section{Problema}% 

Há correlação entre evasão de discentes da Universidade de Brasília e a e seus dados socioeconômicos?
%Há correlação entre evasão de discentes cotistas de baixa renda da Universidade de Brasília e a distância do Campi da sua residência?
%Greve do restaurante universitário é fator preponderante para evasão de discentes cotistas de baixa renda da Universidade de Brasília?
\section{Objetivos}%

\subsection{Objetivo Geral}

O objetivo da proposta é estudar padrões socioeconômicos e comportamentais dos alunos que evadem dos cursos ofertados pela Universidade de Brasília. 
 
\subsection{Objetivo específicos}

\begin{enumerate} 
    \begin{itemize}
        \item   Levantar bibliografia;
        \item   Levantar dados da evasão do ensino superior;
        \item   Levantar dados socioeconômicos disponíveis nas bases de dados dos sistemas da UnB;
        \item	Padronizar a informação de diversas fontes;
        \item	Analisar os dados do período 2008 à 2018;
        \item   Comparar os índices de evasões entre os cursos relacionados a dados acadêmicos e pessoais dos alunos com o intuito  de encontrar padrões que possam auxiliar a identificação de possíveis evasões com antecedência. Permitindo assim um trabalho de mitigação dos riscos de fracasso dos mesmos, ou seja, o aluno sair do curso por um motivo diferente de formatura;
        \item	Apresentar a informação de maneira consistente e única;
        \item	Fomentar as áreas de negócio com informações do estudo para que tomem decisões mais precisas.
        
       % \item   Conhecimento dos dados gerados pelos sistemas acadêmicos legados com enfoque nas evasões.
    \end{itemize}%
\end{enumerate}%

\section{Palavras-chave}%
Evasão no ensino superior; Dados socioeconômicos.


\section{Justificativa}%
Hoje, o Centro de Informática - CPD da Universidade de Brasília - UnB recebe diversas solicitações dos decanatos da Universidade para  extrações de dados das bases acadêmicas que são disponibilizadas através de planilhas para um posterior estudo por parte dos mesmos.

Segundo [Carvalho,2005]%\cite{Carvalho2005DataMining}
, mineração de dados é definida  como o uso de técnicas automáticas de exploração de grandes quantidade de dados de forma a descobrir novos padrões e relações que, devido ao volume de dados, não seriam facilmente descobertas ao olho nu pelo ser humano.

Ao identificar se um aluno de um determinado curso corre o risco de fracassar, por exemplo, abandonar o curso, com antecedência é possível tomar providências cabíveis para mitigar ou até mesmo eliminar este risco. Agindo com antecedência é possível elaborar um plano de intervenção personalizado para melhoria do desempenho evitando o abandono e consequentemente o desperdício de dinheiro público no caso da evasão do mesmo.

O discente quando é selecionado no processo seletivo e ingressa em um curso no ensino superior, na maioria das vezes, têm uma visão de que serão mais bem remunerados. “Assim, para o discente, começar e não terminar um curso de graduação pode acarretar uma frustração profissional que o acompanhará por toda a vida” [Da Cunha,2014].

Visto o exposto, a administração das Universidades podem ser mais eficientes para combater este problema conhecendo melhor os motivos pelos quais levam os discentes a desistirem dos cursos.  
 \begin{comment} 
\section{Metodologia}%

Para realização do estudo serão usadas como fonte as bases de dados dos sistemas de controle acadêmicos da Universidade de Brasília, com enfoque no ciclo de vida acadêmico dos discentes e docentes. Aplicar técnicas de mineração de dados para a descoberta de novos padrões relacionados as diversas formas de evasões dos cursos ofertados pela UnB para que possam prever tendências que  auxiliarão a equipe de planejamento a acompanhar e definir as metas e diretrizes da Universidade no âmbito acadêmico.
\end{comment}

\section{Revisão Teórica}% 

A priori o conceito de evasão parece único, ou seja, o aluno sai do curso por um motivo diferente de formatura.  Porém, o conceito pode mudar de autor para autor. Sendo assim existem vários conceitos para evasão. 

Segundo [Teresa,2001] %\cite{Teresa2001IdentidadeUniversitaria}
, evasão não pode ser considerada somente abandono da instituição. Uma troca de curso, por exemplo, pode ser considerada uma evasão. Contudo, neste projeto abordaremos vários tipos de evasões.
 
Para o [INEP,1996], abandono e evasão são diferentes. “O conceito técnico de abandono é diferente de evasão. Abandono quer dizer que o aluno deixa a escola num ano, mas retorna no ano seguinte. Evasão significa que o aluno sai da escola e não volta mais para o sistema.”.

Neste estudo serão considerados todos os alunos que o motivo de saída do curso não tenha sido formatura. Não entraremos no mérito se o mesmo tenha ou não iniciado o curso, ou seja, se possui uma matrícula serão tratados os motivos de saída do curso. Algo semelhante ao estudo proposto foi realizado por [Pereira,2011], onde o mesmo analisou a dimensão do abandono escolar na Universidade Federal do Mato Grosso e o perfil sócioeconômico dos mesmos. %\cite{Pereira2011EvasaoGrosso}

Foram pesquisadas algumas metodologias e ferramentas,  mas não especificamente com enfoque em sistemas acadêmicos e com fonte de dados em um \textit{data warehouse}. 

Preliminarmente será construído um \textit{data warehouse} utilizando as fonte de dados dos sistemas acadêmicos. [Kimball,2013] %\cite{KimballRalphROSS2011TheModeling}
descreve em seu livro como o próprio título diz, um conjunto de  ferramentas e técnicas para modelagem dimensional. Serve de tutorial para construção de um \textit{data warehouse} para iniciantes e experientes no assunto.

[Hiragi,2008] %\cite{Hiragi2008MineracaoPublicacao}
propôs uma metodologia de mineração de dados, derivada do modelo de referência CRISP-DM, que auxilia a exploração das bases de dados por pesquisadores visando facilitar a realização de tarefas previstas  nas seguintes fases do CRISP-DM: Entendimento do negócio, compreensão dos dados, preparação dos dados, modelagem, avaliação dos modelos gerados e colocação em uso.  Para materializar  a metodologia proposta e automatizar a sua utilização, foi criada uma ferramenta para tal.

Não vamos criar uma ferramenta para implementação de uma metodologia de mineração de dados, mas iremos nos basear no CRISP-DM para coleta e tratamento dos dados.

Com relação às técnicas de mineração de dados, [Han,2006] % \cite{Han2006DataTechniques} 
descreve em seu livro  tudo o que está envolvido no projeto de mineração de dados como bancos de dados relacionais e analíticos, ferramentas de bancos de dados, matemática, estatística, linguagem de programação para banco de dados.











\section{Citações}%
   Citações: 
   
   \cite{Gilioli2016EvasaoDesafios}
   \cite{Catani2006PROUNI:Superior}
   \cite{Andriola2006OpinioesUFC}
   \cite{Lobo2012PanoramaSolucoes}
   \cite{Nassar2011Do.}
   \cite{Pereira2003DeterminantesPara}
   \cite{Silva2014EvasaoTransacao}
   \cite{Veneroso2016AsFederal}
   \cite{Vitelli2010EvasaoFenomeno}
   \cite{Pereira2011EvasaoGrosso}
   \cite{SindicatodasEntidadesMantenedorasdeEstabelecimentosdeEnsinoSuperiornoEstadodeSaoPauloSEMESP2012IndiceInformacao}
   \cite{SilvaFilhoRobertoLealLoboandMotejunasPauloRobertoandHipolitoOscarandLobo2007ABrasileiro.}
   \cite{DaCunhaJacquelineVenerosoAlvesandNascimentoEduardoMendesanddeOliveiraDurso2016RazoesSudeste}
    \cite{Teresa2001IdentidadeUniversitaria}
    \cite{InstitutoNacionaldeEstudosePesquisasEducacionaisInep1996InformeEscolar}
    \cite{Carvalho2005DataMining}
    \cite{KimballRalphROSS2011TheModeling}
    \cite{Hiragi2008MineracaoPublicacao}
\cite{Han2006DataTechniques} 
